\chapter{Abbildungen}
\begin{figure}
    \caption{Die Dokomposition der MOER-Werte von Norwegen (eigene Darstellung)}
    {\includegraphics[width=0.9\textwidth]{\figdir/moer_decomposition_NO}}
    \label{FIG:moer_decomposition_NO}
\end{figure}

\begin{figure}
    \centering
    \subfloat[\centering Positive Einsparungen durch Verschiebungen von Deutschland nach Norwegen]{{\includegraphics[width=.99\textwidth]{\figdir/location-shifting-potential-de_pos} }}%
    \qquad
    \subfloat[\centering Negative Einsparungen durch Verschiebungen von Deutschland nach Norwegen]{{\includegraphics[width=.99\textwidth]{\figdir/location-shifting-potential-de_neg} }}%
    \qquad
    \subfloat[\centering Einsparungen durch Verschiebungen von Norwegen nach Deutschland]{{\includegraphics[width=.99\textwidth]{\figdir/location-shifting-potential-no} }}%
    \caption{Das Potenzial örtlicher Verschiebungen für Deutschland und Norwegen anhand der MOER (eigene Darstellungen)}%
    \label{FIG:location-shifting-potential}%
\end{figure}


\chapter{Tabellen}
\begin{table}[t]
    \centering\small
    \caption{TFT Hyperparameter Tuning}
    \label{TAB:hyperparameter-trained}
    \begin{tabular}{|p{4cm}|p{2cm}|}
    \hline
    attention heads & [1, 4] \\ \hline
    hidden size & [8, 128] \\ \hline
    hidden continous size & [8, 128] \\ \hline
    dropout & [0.1, 0.4] \\ \hline
    max epochs & 10 \\ \hline
    trials & 25 \\ \hline
\end{tabular}

\end{table}

\begin{table}[t]
    \centering\small
    \caption{TFT Verwendete Hyperparameter}
    \label{TAB:hyperparameter-used}
    \begin{tabular}{|p{4cm}|p{1cm}|}
    \hline
    attention heads & 1 \\ \hline
    hidden size & 32 \\ \hline
    hidden continous size & 8 \\ \hline
    dropout & 0.4 \\ \hline
    max epochs & 10 \\ \hline
    learning rate & 0.02 \\ \hline
\end{tabular}

\end{table}


\chapter{Code-Ausschnitte}
\lstinputlisting[language=Python, caption=Python-Code zur Abfrage der MOER-Werte, label=CODE:watttime_get_moer]{\codedir/wattime_get_moer.m}
