\chapter{Abbildungen}\label{CAP:append-figures}
\begin{figure}
    \caption[Dekomposition der MOER Norwegens]{Die Dokomposition der MOER-Werte von Norwegen (eigene Darstellung)}
    {\includegraphics[width=0.8\textwidth]{\figdir/moer_decomposition_NO}}
    \label{FIG:moer_decomposition_NO}
\end{figure}

\begin{figure}
    \centering
    \subfloat[\centering ACF]{{\includegraphics[width=.4\textwidth]{\figdir/acf_moer_no} }}%
    \qquad
    \subfloat[\centering PACF]{{\includegraphics[width=.4\textwidth]{\figdir/pacf_moer_no} }}%
    \caption[ACF und PACF der MOER von Norwegen]{Die ACF und PACF der MOER-Zeitreihe für Norwegen (eigene Darstellung)}%
    \label{FIG:acf_pacf_moer_no}%
\end{figure}

\begin{figure}
    \caption[SARIMAX 168h Vorhersage MOER Deutschland gefiltert]{SARIMAX Vorhersage für 168 Stunden auf der gefilterten MOER des deutschen Stromnetzes (eigene Darstellung)}
    {\includegraphics[width=0.8\textwidth]{\figdir/sarimax_de_168_filtered}}
    \label{FIG:sarimax_de_168_filtered}
\end{figure}
\begin{figure}
    \caption[SARIMAX 72h Vorhersage MOER Deutschland gefiltert]{SARIMAX Vorhersage für 720 Stunden auf der gefilterten MOER des deutschen Stromnetzes (eigene Darstellung)}
    {\includegraphics[width=0.8\textwidth]{\figdir/sarimax_de_720_filtered}}
    \label{FIG:sarimax_de_720_filtered}
\end{figure}

\begin{figure}
    \centering
    \subfloat[\centering Positive Einsparungen durch Verschiebungen von Deutschland nach Norwegen]{{\includegraphics[width=.99\textwidth]{\figdir/location-shifting-potential-de_pos} }}%
    \qquad
    \subfloat[\centering Negative Einsparungen durch Verschiebungen von Deutschland nach Norwegen]{{\includegraphics[width=.99\textwidth]{\figdir/location-shifting-potential-de_neg} }}%
    \qquad
    \subfloat[\centering Einsparungen durch Verschiebungen von Norwegen nach Deutschland]{{\includegraphics[width=.99\textwidth]{\figdir/location-shifting-potential-no} }}%
    \caption[Potenzial örtlicher Verschiebungen zwischen Deutschland und Norwegen]{Das Potenzial örtlicher Verschiebungen für Deutschland und Norwegen anhand der MOER (eigene Darstellungen)}%
    \label{FIG:location-shifting-potential}%
\end{figure}



\chapter{Code-Ausschnitte}
\lstinputlisting[language=Python, caption=Python-Code zur Abfrage der MOER-Werte, label=CODE:watttime_get_moer]{\codedir/wattime_get_moer.m}



\chapter{Informelles Interview}\label{CAP:interview}
Im Rahmen dieser Arbeit wurde ein informelles Interview im Unternehmen Randstad Digital Germany AG durchgeführt, um zu verstehen, wie das Thema nachhaltige Software intern behandelt wird, welche Bedeutung es für die Firma hat und inwiefern Nachhaltigkeit in Kundenanfragen eine Rolle spielt.
Das Interview zielte außerdem darauf ab, praxisnahe Einblicke in die Relevanz und Umsetzbarkeit der in dieser Arbeit vorgeschlagenen Strategien zu erhalten.

Für das Interview wurden zwei Mitarbeiter mit langjähriger Erfahrung und Firmenzugehörigkeit befragt:
\begin{itemize}
    \item \textit{Thomas Frers}, Position: \textit{Senior Solution Architekt}, seit 11 Jahren bei Randstad Digital tätig
    \item \textit{Manuel Fabritius}, Position: \textit{Solution Architekt}, seit 10 Jahren bei Randstad Digital tätig
\end{itemize}
Folgende Fragen wurden gestellt:
\begin{enumerate}
    \item \textit{\glqq In welchem Maß ist das Thema nachhaltige Software in der Randstad Digital Germany AG verankert?\grqq{}} - Ziel dieser Frage war es, die interne Wahrnehmung und Verankerung des Themas zu erfassen.
    \item \textit{\glqq Welche Bedeutung hat Nachhaltigkeit bei den Anfragen von Kunden und potenziellen Kunden von Randstad Digital Germany AG?\grqq{}} - Diese Frage zielte darauf ab, zu verstehen, ob und wie Kunden das Thema Nachhaltigkeit in ihre Anforderungen einfließen lassen.
    \item \textit{\glqq Betrachtest Du die in dieser Arbeit vorgestellten Strategien als umsetzbar im Kundenumfeld? Gibt es konkrete Beispiele aus dem Kundenumfeld, bei denen diese Strategien Anwendung finden könnten?\grqq{}} – Diese Frage sollte Einsichten darüber liefern, ob die theoretischen Ansätze dieser Arbeit in der Praxis Anwendung finden können und welche konkreten Anwendungsbeispiele es bereits gibt oder geben könnte.
\end{enumerate}
Die Interviews wurden in einem offenen Gesprächsformat geführt, um den Befragten die Möglichkeit zu geben, frei über ihre Erfahrungen und Einsichten zu sprechen.
Die gewonnenen Erkenntnisse sind in Abschnitt~\ref{CAP:sustainable-software} und Abschnitt~\ref{CAP:scenarios} eingearbeitet.
Durch diese Einblicke konnte ein praxisnaher Bezug zu den theoretisch erarbeiteten Inhalten hergestellt werden, was zur Relevanz der Forschungsergebnisse beiträgt.
