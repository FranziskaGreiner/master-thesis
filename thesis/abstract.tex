\chapter*{Kurzfassung}
\thispagestyle{empty}
\noindent Diese Masterarbeit befasst sich mit der Entwicklung und Evaluierung von Prognosemodellen zur Bestimmung der marginalen Kohlenstoffintensität von Stromnetzen und untersucht, wie diese Vorhersagen zur Optimierung der Nachhaltigkeit von Softwarearchitekturen beitragen können.
Angesichts der zunehmenden Bedeutung von Energieeffizienz und Reduzierung von Kohlenstoffemissionen in der IT-Branche, bietet diese Arbeit Ansätze zur Emissionsreduzierung von Softwareanwendungen.
Es wurden zwei unterschiedliche Modelle zur Zeitreihenprognose entwickelt und evaluiert, das statistische SARIMAX-Modell und das KI-basierte Modell Temporal Fusion Transformer.

Die Analyse zeigt, dass das TFT-Modell insgesamt präzisere Prognosen der Kohlenstoffintensität liefert.
Ein Vergleich der Stromnetze Norwegens und Deutschlands verdeutlicht, dass die Prognosegenauigkeit stark von der Beschaffenheit des jeweiligen Stromnetzes und dem Verlauf der Kohlenstoffintensität abhängt.
Außerdem sind die Prognosen für eine kürzere Vorhersagelänge präziser, als für längere Vorhersagen.

Der Vergleich der Vorhersagen des norwegischen und des deutschen Stromnetzes macht ersichtlich, dass die Vorhersagegenauigkeit stark mit der Zusammensetzung des Stromnetzes und dem Verlauf der Werte zusammenhängt.
Die Arbeit untersucht die Anwendbarkeit von Lastanpassungs- und Lastverschiebungsstrategien für Softwareanwendungen.
Dabei wird deutlich, dass die gezielte Nutzung von prognostizierten Werten zur marginalen Kohlenstoffintensität, die in diesem Kontext als Indikator für die Umweltverträglichkeit von Strom dient, eine Schlüsselrolle bei der Nachhaltigkeitsoptimierung spielt.
Es werden sowohl die Vorzüge als auch die Herausforderungen dieser Herangehensweise diskutiert und Lösungsansätze für die praktische Umsetzung aufgezeigt.

Die Ergebnisse dieser Arbeit sind besonders relevant für Entscheidungsträger und Entwickler in der IT-Branche, die nachhaltige Lösungen anstreben.
Sie bieten wertvolle Einsichten in die Effektivität von Prognosemodellen und die strategische Lastverlagerung, mit dem Ziel, den Energieverbrauch und die Kohlenstoffemissionen von Rechenzentren und Softwareanwendungen zu minimieren.
% Diese Erkenntnisse eröffnen neue Wege für die Anwendung von Optimierungsstrategien wie Scheduling, Scaling, Demand Shifting und Demand Shaping, um Softwareanwendungen nachhaltiger zu gestalten.
% Durch die Integration von Zeitreihenprognosemodellen in die Softwareentwicklung wird ein wichtiger Beitrag zur Reduzierung der Umweltauswirkungen der IT-Industrie geleistet.

\bigskip

\noindent
\textbf{Schlagworte:} Nachhaltige Softwareanwendungen, Kohlenstoffintensität, Zeitreihenvorhersage, Temporal Fusion Transformer, SARIMAX

\bigskip
\noindent
\textbf{Repository:}
\url{https://github.com/FranziskaGreiner/master-thesis.git}

