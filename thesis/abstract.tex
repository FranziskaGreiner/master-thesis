\chapter*{Kurzfassung}
\thispagestyle{empty}
Diese Masterarbeit befasst sich mit der Entwicklung und Evaluierung von Prognosemodellen zur Bestimmung der marginalen Kohlenstoffintensität von Stromnetzen und untersucht, wie diese Vorhersagen zur Optimierung der Nachhaltigkeit von Softwarearchitekturen beitragen können.
Angesichts der zunehmenden Bedeutung von Energieeffizienz und Reduzierung von Kohlenstoffemissionen in der IT-Branche, bietet diese Arbeit Ansätze zur Emissionsreduzierung von Softwareanwendungen.
Es wurden zwei unterschiedliche Modelle zur Zeitreihenprognose entwickelt und evaluiert, das statistische SARIMAX-Modell und das KI-basierte Modell Temporal Fusion Transformer.
Die Ergebnisse zeigen, dass das TFT-Modell insgesamt eine präzisere Vorhersage der Kohlenstoffintensität ermöglicht.
Der Vergleich der Vorhersagen des norwegischen und des deutschen Stromnetzes macht ersichtlich, dass die Vorhersagegenauigkeit stark mit der Zusammensetzung des Stromnetzes und dem Verlauf der Werte zusammenhängt.
Diese Erkenntnisse eröffnen neue Wege für die Anwendung von Optimierungsstrategien wie Scheduling, Scaling, Demand Shifting und Demand Shaping, um Softwareanwendungen nachhaltiger zu gestalten.
% Durch die Integration von Zeitreihenprognosemodellen in die Softwareentwicklung wird ein wichtiger Beitrag zur Reduzierung der Umweltauswirkungen der IT-Industrie geleistet.

\bigskip

\noindent
Schlagworte: Nachhaltige Softwareanwendungen, Kohlenstoffintensität, Zeitreihenvorhersage, Temporal Fusion Transformer, SARIMAX

